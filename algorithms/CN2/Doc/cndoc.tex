% "@(#)cndoc.tex 4.8 6/15/92 MLT"
\documentstyle[mlt,named]{esprit_report}
\def\gap{\hspace{1em}}
\def\CN{{\sc cn2}}
\def\AQ{{\sc aq}}
\def\ID{{\small NewID}}
\def\tab{{\sc tab}}
\def\esc{{\sc esc}}
\def\und{\mbox{\_}}
\def\emacs{{\sc emacs}}
\def\ret{\raisebox{-0.75ex}{$\swarrow$}}
\newcommand{\bt}{~\mbox{${\bullet}$~}}          % a bullet
\newcommand{\m}[1]{\mbox{${\tt #1}$}}           % nice maths mode
\documenttype{Manual}
\author{Robin Boswell \nl The Turing Institute Limited}
\date{January 1990}
\title{Manual for CN2 version 6.1}
\reference{TI/P2154/RAB/4/1.5}
\task{T}
\doctype{Manual}
\status{Draft}
\classification{Public}
\docclass{ITM}
\intid{TI/MLT/4.0T/RAB/1.2}
\distribution{Universal}
\begin{document}
%\makefrontpage
\maketitle
\maketable

\begin{abstract}
This document is an introduction and user's manual for release 6.1
of the rule-induction program \CN.
\end{abstract}
\newpage

\tableofcontents
\newpage

\section{Introduction}
The program described in this manual is version 6.1
of the Turing Institute's implementation of \CN.  For details of
the \CN\ algorithm itself, and plans for further work, see
\cite{mlt:7}.  

   For brevity, I shall use the term ``\CN'' in the following account
to refer both to the algorithm and to its current implementation. 

\section{Overview}
For instructions on installing \CN, see Appendix
\S\ref{getting_started}.

\CN\ is a rule-induction program which takes a set of {\it examples}
(vectors of attribute-values), and generates a set of {\it rules} for
classifying them.     It also allows you to {\it evaluate}
the accuracy of a set of rules (in terms of a set of pre-classified
examples).

To use the program, you must provide an 
ASCII file of examples in a standard format (described in 
\S~\ref{input}). When the program has induced some rules from these
examples, you may display or evaluate them, or store them in a file.
Rules are stored in a human-readable form, but they may
also be read back in by \CN, to save re-calculating them.

If you have data files in MLT CKRL format, then these
can be converted into \CN's format by 
a CKRL translator (provided by the Turing Institute).
In addition, \CN\ can itself translate attributes, examples
and decision-trees into CKRL format (see the {\bf Ckrl} command
in section~\ref{commands}).

   See Appendix \S\ref{script} for an example of a terminal
session with \CN.

   The interface to \CN\ is similar to that for the Turing Institute's
other recently-developed program \ID\, described in \cite{mlt:10}, so 
if you are 
using both programs, you may find the following comparisons useful:

\begin{description}
 \item[Input]  The format of attribute and example files is
  identical for the two programs.  (Declarations of attributes
  as `binary' are meaningful only for \ID: such declarations will 
  be silently ignored by \CN.)

 \item[Commands] The interfaces are similar, but \ID\ has a smaller
   menu of commands
  than \CN, because it has fewer customisable parameters
   (such as `significance threshold').

 \item[Output] The output formats differ (as do the algorithms):
 \CN\ generates production rules and \ID\ generates decision trees.
\end{description}

\section{Example and Attribute Files}
\label{input}

The input to \CN\ consists of:
\begin{enumerate}
 \item A set of attribute declarations.
 \item A set of examples.
\end{enumerate}

Items 1.\ and 2.\ may be presented in the same file, or in two different
files.  Storing the attributes separately from the examples makes 
it possible to read and process several example files in succession
without having to re-load the attribute declarations (provided all
the examples use the same attributes).

\subsection{Files}
Input files are of three types:
\begin{enumerate}
 \item Attribute files.
 \item Examples files.
 \item Attribute and Example files.
\end{enumerate}

The most complex of these, the attribute and example file, 
consists of a header, followed by attribute declarations,
then examples.  The two types of data are separated by punctuation,
as specified in \S~\ref{att_syntax}.
(See \S~\ref{example} for a specimen of
such a file).

Attribute and example files consist of the 
appropriate header followed respectively by attributes and examples.

In all file types, characters between a `\%' and the next end-of-line 
are regarded as comments and ignored.

Formally: 
\begin{quotation} \sf
\begin{tabbing}
foo \= qq \= wumpus wumpus wumpus wumpus wump \= \kill
attribute-and-example-file ::= \\
\>\>   att-and-ex-header \gap attribute-delarations \gap
                          separator \gap examples \gap separator \\[1.5ex]
att-and-ex-header ::=  \\
\>\>   ``$\ast\ast$ATTRIBUTE AND EXAMPLE FILE$\ast\ast$'' \\[1.5ex]
separator ::=  \\
\>\>   ``\verb+@+'' \\[2.5ex]
attribute-file ::= \\
\>\>   att-header \gap attributes \\[1.5ex]
att-header ::= \\
\>\>   ``$\ast\ast$ATTRIBUTE FILE$\ast\ast$'' \\[2.5ex]
example-file ::= \\
\>\>   ex-header \gap examples \\[1.5ex]
ex-header ::=  \\
\>\>   ``$\ast\ast$EXAMPLE FILE$\ast\ast$''
\end{tabbing}
\end{quotation}
The format for {\it attributes} and {\it examples} in the above
grammar is defined below.

\subsection{Attributes}
\subsubsection{Semantics}
Attributes are of two types: discrete and ordered.  A discrete 
attribute takes one of a finite set of values (and the set of values
has no further structure).  An ordered attribute takes numeric
values: integer or floating point.

An attribute {\it declaration} consists of the name of the attribute,
together with either:
\begin{itemize}
 \item Its possible values (for a discrete attribute), or
 \item Its numeric type (for an ordered attribute).
\end{itemize}

For the purposes of the rule-induction algorithm, one of the
attributes is distinguished as the {\it class attribute}.
The aim of the algorithm is to find rules whereby the class
attribute of an example may be inferred from the non-class 
attributes. The class attribute must be discrete.


\subsubsection{Attribute ordering}
In some domains, it may be desirable to impose constraints
on the order in which attributes are tested.  For example,
one should determine whether a patient is female before asking
whether she is pregnant, rather than afterwards.  Such
constraints may be imposed on \CN\ by means of 
{\it attribute ordering declarations}, which (optionally) follow
the attribute declarations. 

Note that it is not possible
to make the {\it result} of one attribute test a precondition 
of some other test (for example, ``if sex is male, then don't
ask about pregnancy'') but given sensible data, an inappropriate
test will yield a low entropy gain, and so be excluded anyway
by \CN's existing criteria.

An attribute ordering declaration takes the form:
\begin{quotation}
Attribute$_1$ BEFORE Attribute$_2$;
\end{quotation}

You may include as many such declarations as you wish.  Note
that \CN\ does not currently check for ``loops'' (A before B
before C before A \ldots): if you introduce such a loop, the
result will be that none of the attributes involved will ever
be used.

Finally, note that the order in which attribute tests are {\it displayed}
in \CN's output has no significance (as it happens, it matches the
order of attribute declaration), so in particular is not affected by
attribute ordering constraints.  The only effect of, for example, the constraint 
{\tt sex BEFORE pregnancy} is to ensure that any
rule containing a test of {\tt pregnancy} also contains a test of 
{\tt sex}.

\subsubsection{Syntax}
\label{att_syntax}
An attribute declaration consists of the attribute's name, followed
either by its possible values (if it's discrete), or its numeric
type (if it's ordered), separated by appropriate punctuation.
The class attribute must be declared last.  (i.e.\ \CN\ will
treat the last attribute as the class attribute).

Formally:
\begin{quotation} \sf
\begin{tabbing} 
foo \= qq \= wumpus wumpus wumpus wumpus wump \= \kill
attribute-data ::= \\
\>\> attribute-declarations \gap att-ordering-section\\
\> $\mid$ \> attribute-declarations \\[2.5ex]
attribute-declarations ::= \\
\>\>attribute-declaration \\
\> $\mid$ \> attribute-declarations \gap attribute-declaration \\[2.5ex]
attribute-declaration ::= \\
\>\>        string ``:'' attribute-values ``;'' \>   \rm \% For discrete attributes \\
\>$\mid$ \> string ``:'' ``('' type ``)''      \> 
                           \rm \% For ordered attributes \\[2.5ex]
attribute-values ::=  \\
\>\>      string \gap attribute-values  \\
\>  $\mid$  \> string  \\[2.5ex]
type ::= \\
\>\>   ``FLOAT''   \\
\> $\mid$ \> ``INT'' \\[2.5ex]
string ::=  \\
\>\>  quoted-string   \\
\>$\mid$ \> unquoted-string  \\
att-orderings-section ::= \\
\>\> ``ORDERINGS'' attribute-orderings\\[2.5ex]
attribute-orderings ::= \\
\>\> attribute-ordering \\
\>  $\mid$  \> attribute-orderings attribute-ordering \\[2.5ex]
attribute-ordering ::= \\
\>\> string ``BEFORE'' string ``;'' \\
\end{tabbing}
\rm
An {\it unquoted-string} is a sequence of one or more characters made
up of upper
or lower-case letters, digits, and the characters ``--'', ``+'' and ``\und'',
with the constraint that the first character must be a letter or ``\und''.

A {\it quoted-string} is any sequence of printable characters (except
newline and the double-quote character) surrounded by double-quotes.

Thus the following are all valid strings:
\begin{quotation}
\begin{verbatim}
Fred   _42  "42"  "%^=- !"
\end{verbatim}
\end{quotation}
and the following are not valid strings:
\begin{quotation}
\begin{verbatim}
42   -green  + 
\end{verbatim}
\end{quotation}
\end{quotation}

For example:

\begin{verbatim}
   sex: male female;
   job: cleaner secretary manager director hacker;
   salary: (FLOAT)
   status: content discontented   % The class attribute

   ORDERING

   sex BEFORE salary;

\end{verbatim}

\subsection{Examples}
\subsubsection{Semantics} 
Each example consists of a set of values, one for each attribute,
and may also include a weight.

\paragraph{Values}
Each value must be of the appropriate type.  However, in addition
to the numeric and string values given in the attribute declarations,
any attribute except the class attribute may take
one of the two special values ``Unknown'' or ``Don't Care''.

The ``Unknown'' value
is used to represent an unknown value (!), such as frequently 
occur in real-world data.
The ``Don't Care'' value, on the other hand, is assigned to
an attribute whose value is irrelevant to the classification 
of the example.  In practice, ``Don't Care'' values are
most likely to arise as a means of compressing synthetic data.

Thus, an ``Unknown'' attribute corresponds roughly to
an existentially quantified variable,
whereas a ``Don't Care'' attribute is universally quantified.

A brief account of how \ID\ handles ``unknown'' and ``don't care''
values appears in Appendix \ref{unknown}

\paragraph{Weights}
By default, each example is assigned a weight of 1, and this is
sufficient for most applications.  However, in some
cases it may be useful to assign different weights to examples; the 
weight must be a positive real number.  If you use this facility, you
should note that references in this document to the ``number of
examples'' in a set (e.g.\ \ref{all_eval}) really mean ``the sum of the weights
of the examples''.


\subsubsection{Syntax}

\begin{quotation} \sf
\begin{tabbing}
foo \= qq \= wumpus  wumpus w \= \kill
examples ::= \\
\>\>  example \\
\> $\mid$ \> examples \gap example  \\[2.5ex]
example ::= \\
\>\>   values ``;''  \\
\>$\mid$ \>values ``w'' number \\[2.5ex]
values ::=  \\
\>\>   value  \\
\>$\mid$ \>values \gap value \\[2.5ex]
value ::=    \\
\>\>string   \\
\>$\mid$ \> number  \\
\>$\mid$ \> ``?''  \> \rm \% Unknown \\
\>$\mid$ \> ``$\ast$'' \> \rm \% Don't Care \\[2.5ex]
number ::= \\
\>\> integer \\
\>$\mid$ \> float \\
\end{tabbing}
\end{quotation}

In addition, the order of items in {\it values} must correspond to the
order of the attributes as previously declared, so the length
of {\it values} must be equal to the number of attributes, and 
each {\it value$_n$} must match the type of {\it attribute$_n$}.

\subsection{Specimen input file}
\label{example}
This is a small ``Attribute and Example'' file:

{\small
\begin{verbatim}
**ATTRIBUTE AND EXAMPLE FILE**
% Vertebrate classification
skin_covering: none hair feathers scales;
milk:          yes no;
homeothermic:  yes no;
habitat:       land sea air;
reproduction:  oviparous viviparous;
breathing:     lungs gills;
class:	mammal fish reptile bird amphibian;

@
% mammal
hair           yes     yes     land    viviparous      lungs mammal;
none           yes     yes     sea     viviparous      lungs mammal;
hair           yes     yes     sea     oviparous       lungs mammal;
hair           yes     yes     air     viviparous      lungs mammal;

% fish 
scales         no      no       sea    oviparous       gills fish w 4;

% reptile
scales         no      no      land    oviparous       lungs reptile w 3.2;
scales         no      no      sea     oviparous       lungs reptile;

% bird
feathers       no      yes     air     oviparous       lungs bird;
feathers       no      yes     land    oviparous       lungs bird;

% amphibian
none           no      no      land    oviparous       lungs amphibian;
@
\end{verbatim}}


\section{Rules: files and  evaluation}

\subsection{File format}
When \CN\ writes a rule set to file, it does so in a
human-readable ASCII form, but unlike example and attribute files,
you are not expected to write or modify rule files yourself.
(i.e.\ you do so at your own risk, and should note that
\CN\ does very little error-checking when reading rule files).
In future releases, a graphical interface may be provided
for manipulating rules.

  A small rule-set appears as part of the trace on 
page~\pageref{little_rules}.

\subsection{Class distributions}
\label{distributions}
The list of numbers at the end of each rule indicates the 
number of training examples covered by that rule, divided into
classes.

The precise significance of the counts depends on whether
the rules are ordered or unordered.  Ordered rules are
intended to be executed in order (!), so the counts 
associated with rule $N$ refer to the examples 
covered\footnote{A rule is said to {\it cover} an example if the 
example satisfies the conditions of the rule}
by rule $N$ which were $not$ covered by any of the rules
$1$ through $N-1$.  This applies also to the default rule---since
this rule has no conditions, its counts comprise all
the examples which were not covered by the preceding rules.

The counts associated with each of a set of {\it unordered}
rules, however, comprise all the examples covered by that rule
including those which may be covered by other rules as well.
Again, this applies equally to the default rule, whose count
therefore comprises the whole example set.

\subsubsection{Rare classes}
Given the above definition of ``class distribution'', you
might expect that the class predicted by a rule would always
show the largest class-count in the corresponding distribution; 
so to avoid possible confusion when using the program, you should
note that this won't always be the case.

In the case of ordered rules, the class predicted by each rule 
will be the majority class among the examples covered, 
{\it by definition}.

In the case of unordered rules, however, this property will not
always hold, though it usually will.  The exceptions are rules
which give better than average predictions of rarely-occurring
classes.  For example, in a medical domain, where the program 
is trying to predict the rarely-occurring phenomenon of 
diabetes, the following rule might be induced:


\begin{tabbing}
foo \= h \= hoo ho\= \kill
\> IF \>\>     temperature = high \\
\> \>  AND \>   fluid-intake = high  \\
\> THEN \>\>   diagnosis = diabetes [420 0 0 \underline{126}]
\end{tabbing}


where there are 126 examples of diabetes, and 420 of something
different (colds, as it happens).  Since the occurrence of
diabetes in the general population is only 0.1\%, a rule
that predicts it with an accuracy of 30\% is clearly of some
use, even if 70\% of patients satisfying the conditions just
have colds.



\subsection{Evaluation}
\label{evaluation}
The evaluation module takes a set of rules and a pre-classified set of 
examples, and compares the classification given by the rules
with the given class-values.  There are currently two modes of
evaluation: {\it all} and {\it individual}.  (Individual rule 
evaluation is a recent addition in response to local demand).

\subsubsection{``All'' rule evaluation}
\label{all_eval}
In this mode, the evaluation is of a set of rules taken as a whole.
The results are displayed in the form of a matrix:

\begin{tabular}{rr|rrrrr|r} 
\multicolumn{3}{r}{PREDICTED} \\
\multicolumn{2}{l|}{ACTUAL} & 
mammal &  fish &  reptil  & bird  &  amphib & Accuracy \\ \cline{2-8}
\raisebox{1.5ex}{\rule{0mm}{1ex}}    
&  mammal &  3 &  0 &  0 &  0 &  0 &   100 \%  \\
  
&    fish &  0 &  1 &  0 &  0 &  0 &   100 \%   \\

& reptile &  0 &  0 &  1 &  0 &  0 &   100 \%  \\

&    bird &  1 &  0 &  1 &  2 &  0 &  50 \%   \\

& amphibia &  0 &  0 &  0 &  0 &  1 &  100 \% \\
\multicolumn{8}{l}{} \\
\multicolumn{8}{l}{Overall accuracy: 80 \%} \\
\multicolumn{8}{l}{Default accuracy: 40 \%}
\end{tabular}

The entry in row $i$, column $j$ of the matrix is the number of
examples classified by the rules  as class $j$ which were really of 
class $i$. (So, in the sample matrix above, 
there was 1 example which the rules classified as a mammal, but
which was really a bird.)  Fractional values may arise if the examples
include ``Unknown'' or ``Don't Care'' values.

The ``Default accuracy'' is only applicable to unordered
rule sets; it is the accuracy resulting from classifying
all examples according to the majority class (which is what
the default rule does).

\subsubsection{``Individual'' rule evaluation}
This mode of evaluation is currently available only for unordered
rule sets, though a similar facility for ordered rule sets may be
provided in future. 

For each rule, a matrix similar to the above is calculated, but
in this case, the class values are reduced to ``class predicted by
rule'' and ``other classes''.  For example:

\begin{tabbing}
fred \= h \= hoooo \= \kill
\> --------------- Rule 2 --------------- \\
\>IF \>\>   7.50 $<$ number\_of\_legs $<$ 25.00   \\
\>THEN \>\> class = spider  [0.25 2 0 0]
\end{tabbing}

\begin{tabular}{rr|rr|r} 
\multicolumn{4}{r}{FIRED?} \\
\multicolumn{2}{l|}{ACTUAL CLASS} & 
             Yes  &   No &    Accuracy \\ \cline{2-5}
\raisebox{1.5ex}{\rule{0mm}{1ex}}    
&    spider  &  2.00  &  0.00  &  100.0 \% \\
& Not spider &  0.25  &  7.75  &   96.9 \% \\
\multicolumn{4}{l}{} \\
\multicolumn{4}{l}{Overall accuracy: 97.5 \%}
\end{tabular}



\subsection{Inducing and testing rules}
To create and  test a set of rules, you will need to carry out the 
following operations (the actual commands required are described
in the next section).

\begin{enumerate}
 \item Load some attributes and examples
 \item Induce some rules
 \item Load some more examples
 \item Evaluate the rules
\end{enumerate}

In future releases, I shall probably provide commands for 
(e.g.)\ partitioning example sets, and selecting sub-sets, within
the program, thus simplifying the above  sequence.


\section{Using CN}
\subsection{Interactive use}
\label{interaction}

   Note: in specimens of terminal interaction user input appears
underlined,  and ``\ret'' represents the use of the `Return' key. For example:

\begin{quotation} \par \noindent \tt
READ$>$ Both, Atts, or Examples? \underline{B}oth  \\
READ$>$ Filename?
\underline{\mbox{\rule[-.1cm]{0cm}{0.2cm}Data/animals}} \ret
\end{quotation}

In the first line, the user typed the single character `B' (and did
{\it not} hit the return key), and the 
program expanded the command to `Both'.  In the second line,
the user typed the file name `Data/animals', and then hit `Return'.

When you start \CN, you will see this friendly greeting:
\vspace{-2ex}
\begin{quotation} \par \noindent 
\small \tt
\begin{tabbing}
foo wumpus \=     ********* \= ******************** \= * \kill
\>            ******************************** \\
\>            *  \>                      \>  * \\
\>            *  \>    Welcome to CN2!   \>  * \\
\>            *  \>                      \>  * \\
\>            ******************************** \\
\end{tabbing}
\end{quotation}
\vspace{-2ex}
followed by the prompt:
\vspace{-2ex}
\begin{quotation} \tt
CN$>$ 
\end{quotation}
\vspace{-2ex}
This is the {\it top-level} prompt.  If you now type `h' or `?' (for help), 
the program will display a list of the commands that may be entered
at this point,
and then re-display the top-level prompt, ready for another command.

Some operations require several commands to perform.  For example,
if you want \CN\ to read in a file, you have to tell it what
sort of thing the file contains, and the name of the file.  
In this case, your initial `read' command will cause \CN\ to
change from top-level mode to file-reading mode, and it will
display a different prompt, thus:
\vspace{-2ex}
\begin{quotation} \tt
READ$>$ 
\end{quotation}
\vspace{-2ex}
When the file-reading has been completed, \CN\ redisplays its
top-level prompt, so the whole sequence might appear as follows:

\begin{quotation} \par \noindent \tt
CN$>$ \underline{R}ead \\
READ$>$ Both, Atts, or Examples? \underline{B}oth \\
READ$>$ Filename? \underline{\mbox{\rule[-.1cm]{0cm}{0.2cm}Data/animals}} \ret \\
Reading attributes and examples \ldots \\
10 examples! \\
Finished reading attribute and example file! \\
CN$>$ 
\end{quotation}

\subsubsection{Top-level commands}
\label{commands}
Note that this section assumes an understanding of the \CN\ algorithm
as specified in \cite{mlt:7}, and you should refer
to this document if you are unsure of the meaning of ``star size'',
``significance threshold'',  ``ordered rules'', etc.

Each of these commands is invoked by typing its initial letter (upper
or lower case), except for ``Execute'', which is invoked by ``x''.

\begin{description}
\item[Read:] Read an attribute, example, attribute-and-example or rule
file.  When loading several files in succession, you should bear in
mind that \CN\ can only retain in memory one set of attributes, and 
one set of examples, at any one time.  Consequently:

  \begin{enumerate}
   \item Before loading an example or rule file, you must load the appropriate
         attributes.
   \item Loading a file of any type overwrites any data of that type
         which you may have loaded previously (even if the load 
         fails---for example, because of a syntax error in the new file).
   \item In addition, loading a set of attributes causes any previously-loaded
         examples to be lost.  
  \end{enumerate}

   The entry of file-names is facilitated by the following features:
  \begin{enumerate}
    \item As in \emacs, ``$<$\tab$>$'' invokes file-name completion.  If
         the prefix you have typed
        is ambiguous, the program will show you a list of
         the possible completions.

    \item As in \emacs, ``$<$\esc$>$ h'' is bound to
           ``delete-word-backwards''.
  
    \item
    Each time you type a `/' as part of a UNIX path-name, the
    program checks that the directory referred to exists and is readable.
     If not, it will prevent you from typing further (you must delete
    back and correct the invalid path).
 \end{enumerate}

\item[Induce:] Run \CN\ on the most recently read set
of examples.  The behaviour of \CN\ can be modified by changing
various parameters (see the commands below).  However, all these
parameters have default values set at start-up, so there is no
need to set them before processing data.

\item[Write:] Write a set of rules to file.  (You must first
have read in some examples and run \CN\ on them). File-name entry works
as for {\bf Read}, above.

\item[Ckrl:] Write data in CKRL format. Either the current set of
    attributes and examples (together) or the current decision-tree
    may be written.  Note that the former option allows \CN\ to be used
    as a translator from \CN\ format to CKRL.

\item[eXecute:] Enter ``evaluation'' mode, in which the performance
                of the current rules may be assessed with respect to
                the current examples.  (See \S~\ref{evaluation}).
                In evaluation mode, valid commands are:

   \begin{description}
    \item[All:] Evaluate the set as a whole.
    \item[Individual:] Evaluate rules individually.
    \item[Help/?:]  Display a menu of commands.
    \item[$<$RET$>$:]  Return to the {\bf CN$>$} prompt.
   \end{description}


\item[Algorithm:] Specify whether \CN\ is to produce 
  ordered or unordered sets of rules.

\item[Error estimate:] Specify whether \CN\ is to use the Laplacian
   or the na\"{\i}ve estimate to assess the accuracy of a rule.

   The formulae for the two estimates are as follows:

\begin{tabular}{ll}
 {\rm Laplacian:}  &  \#{\rm Correct}$ + 1$ / 
                                 \#{\rm Examples} $+$ \#{\rm Classes} \\
 {\rm Na\"{\i}ve:} &  \#{\rm Correct} / \#{\rm Examples}
\end{tabular}


   Preciesly which example sets these formulae are applied to
depends on whether the rule set being induced is ordered 
or unordered (see the discussion of ``counts'' in  \S~\ref{distributions}).

\item[Star size:] Query or alter the star size.

\item[Threshold:] Query or alter the significance threshold.

\item[Display:] Provide further information about the search.  
By default, the program just displays each rule as it finds it, but
if required it can, for example, indicate whenever a new ``best node''
is found, or which node is currently being specialised.  
(Type `h' at the  ``SET TRACE FLAGS$>$'' prompt for a list of options.)

This facility
can be used to clarify what the algorithm is doing, particularly
if it is generating unexpected answers.  

\item[Help/?:]  Display a menu of commands.

\item[Quit:]  Quit
\end{description}


\subsection{Non-interactive use}
It is possible to run \CN\ non-interactively by supplying it
with a sequence of commands in a file.  In this case, \CN\ will write
a trace of its activities to the standard output, so if you want 
to run it in the background, you should redirect its output
either to a log file or to /dev/null.  For example:

\vspace{-3ex}
\begin{quotation} \tt
\%\ cn $<$ cn.commands $>$ cn.log \&
\end{quotation}


Commands should be listed in the command file more or less as they 
would be typed in answer to \CN's prompt.  In the case of 
single-character commands, the complete command may be used, to
make the file more readable (the program will ignore all but the first
character of the word): each such command should be separated from
the next item by some form of whitespace, but not necessarily
a newline.  Filenames must be followed by a newline. 

Characters from `\#' to the next end-of-line will be regarded as
comments and ignored (c.f. shell scripts).

For example:

\vspace{-6ex}
\begin{quotation} \small \par \noindent
\begin{verbatim}
read atts Data/animals.atts
read exs Data/animals.exs
alg unordered 
star 7
display none quit
induce
write
Data/animals.rules
quit
\end{verbatim}
\end{quotation}
\vspace{-2ex}

\bibliography{mltdoc}
\bibliographystyle{named}
  

\appendix
\section{Getting Started}
\label{getting_started}
To read the contents of your release tape, load the tape into your
tape-drive, {\bf cd} to the 
directory where you want the sources and data to be installed,
and type:

\begin{quotation}
\bf     tar xvf {\it tape-device-name}
\end{quotation}

   This should create a directory {\bf Release}, with
sub-directories {\bf NewId}, {\bf CN2}, {\bf Simple\_Data}, 
and {\bf Docs}.

Next, {\bf cd} to {\bf Release/CN2}, where you should find the
sources for \CN, and a {\bf Makefile}.

  If you are using SunOS 3, then you will need to change
the definition of ``OS'' in ``mdep.h''. Replace the line:

\vspace{-3ex}
\begin{quotation}
\verb+#define OS (4)+
\end{quotation}
\vspace{-3ex}

with 

\vspace{-3ex}
\begin{quotation}
\verb+#define OS (3)+
\end{quotation}
\vspace{-3ex}

Now type:

\vspace{-3ex}
\begin{quotation}
\verb+make cn+
\end{quotation}
\vspace{-3ex}

This will cause {\bf cn} to be compiled.

You will find some example and attribute files in the directory
{\bf Release/Simple\_Data}.


\section{Example run}
\label{script}

This section records a terminal session in which \CN\ is applied
to a small example file called `animals'.  This file has been
included on your release tape, so that you can experiment with
it without having to type it in. 

As in \S~\ref{interaction}, user input is underlined, and
carriage-return denoted by \ret.

\subsection*{Points to Note}
\begin{itemize}
 \item You will notice that in response to the first `Induce' command, 
\CN\ prints out the resulting rule set twice.  This is because,
at the default level of tracing, the program prints out each
rule as it finds it, and then prints out the complete set
of rules when it has finished.   In the case of unordered rules,
the rules are not accompanied by class distributions when
first calculated, since it is more convenient to delay calculation
of these distributions until the rule-set is complete.

\S~\ref{distributions} discusses the precise meaning of the class
distributions.

 \item Note how raising the significance threshold from 1 to 10
       for the second ``run''
       gives smaller but less accurate  rules.
\end{itemize}


\vspace{-2ex}
\begin{quotation}  \par \noindent
\small \tt
\begin{tabbing}
foo wumpus \=     ********* \= ******************** \= * \kill
\>            ******************************** \\
\>            *  \>                      \>  * \\
\>            *  \>    Welcome to CN2!   \>  * \\
\>            *  \>                      \>  * \\
\>            ******************************** \\
\end{tabbing}
CN$>$ \underline{R}ead \\
READ$>$ Both, Atts, or Examples? \underline{B}oth \\
READ$>$ Filename? \underline{\mbox{\rule[-.1cm]{0cm}{0.2cm}Data/animals}} \ret \\
Reading attributes and examples \ldots \\
10 examples! \\
Finished reading attribute and example file! \\
CN$>$ \underline{A}lgorithm   \\
Algorithm is currently set to UN\_ORDERED \\
ALGORITHM$>$ \underline{O}rdered \\
  CN will produce an ordered rule set \\
CN$>$ \underline{I}nduce \\
Running CN on current example set\ldots \\[-4.5ex]
\begin{tabbing}
h \= hoo \= \kill
Best rule is: \\              
IF \>\>   milk = yes \\
THEN \>\>  class = mammal   [4 0 0 0 0] \\
Best rule is: \\
IF \>\>    skin\_covering = feathers \\
THEN \>\>  class = bird   [0 0 0 2 0] \\
Best rule is: \\
IF \>\>    skin\_covering = scales \\
\>  AND \> breathing = lungs \\
THEN \>\>  class = reptile   [0 0 2 0 0] \\
Best rule is: \\
IF \>\>    skin\_covering = none \\
THEN \>\>  class = amphibian   [0 0 0 0 1] \\
Best rule is: \\
ELSE (DEFAULT) class = fish  [0 1 0 0 0 0] \\
\end{tabbing}
\begin{tabbing}
foo wumpus \=     *----- \= -------------------------- \= * \kill
\>            *--------------------------------* \\
\>            $\mid$ \>                      \>  $\mid$\\
\>            $\mid$ \> \raisebox{1.5ex}{ORDERED RULE LIST}  \>  $\mid$\\
\>            *--------------------------------* \\
\end{tabbing}
\begin{tabbing}
h \= hoo \= \kill
IF \>\>    milk = yes \\
THEN \>\>  class = mammal   [4 0 0 0 0] \\
ELSE \\
IF \>\>    skin\_covering = feathers \\
THEN \>\>  class = bird    [0 0 0 2 0] \\
ELSE \\
IF \>\>    skin\_covering = scales \\
\>  AND \> breathing = lungs \\
THEN \>\>  class = reptile    [0 0 2 0 0] \\
ELSE \\
IF \>\>    skin\_covering = none \\
THEN \>\>  class = amphibian   [0 0 0 0 1] \\
ELSE \\
(DEFAULT) class = fish   [0 1 0 0 0] \\
\end{tabbing}
\vspace{-5ex}
CN$>$ \underline{A}lgorithm   \\
Algorithm is currently set to ORDERED \\
ALGORITHM$>$ \underline{U}nordered \\
  CN will produce an unordered rule set \\
CN$>$ \underline{T}hreshold \\
Current threshold is 1.00 \\
New threshold: \underline{\mbox{\rule[-.1cm]{0cm}{0.2cm}7}} \ret \\
OK! \\
CN$>$ \underline{D}isplay \\
SET TRACE FLAGS$>$ \underline{N}one \\
  All tracing is now turned OFF \\
SET TRACE FLAGS$>$ \ret \\
CN$>$ \underline{I}nduce \\
Running CN on current example set\ldots \\
\begin{tabbing}
foo wumpus \=     *----- \= ----------------------------- \= * \kill
\>         *-----------------------------------* \\
\>         $\mid$ \>                      \>  $\mid$ \\
\>         $\mid$ \> \raisebox{1.5ex}{UN-ORDERED RULE LIST}  \> $\mid$ \\
\>         *-----------------------------------* \\
\end{tabbing}
\begin{tabbing}
h \= hoo \= \kill
IF \>\>    milk = yes \\
THEN \>\>  class = mammal  [4 0 0 0 0]  \\[2.5ex]
IF \>\>    skin\_covering = scales \label{little_rules}  \\ 
THEN \>\>  class = fish  [0 1 2 0 0] \\[2.5ex]
IF \>\>    skin\_covering = scales \\
THEN \>\>  class = reptile  [0 1 2 0 0] \\[2.5ex]
IF \>\>    homeothermic = no \\
\>  AND \> breathing = lungs \\
THEN \>\>  class = amphibian  [0 0 2 0 1] \\[2.5ex]
ELSE (DEFAULT) class = mammal  [4 1 2 2 1] \\
\end{tabbing}
\vspace{-5ex}
CN$>$ \underline{Q}uit \\
Have a nice day! \\
\end{quotation}

\section{Handling of Unknowns and Don't Cares}
\label{unknown}
\subsection{Training (Learning) Phase}

During rule generation, a similar policy of handling unknowns and don't
cares is followed: unknowns are split into fractional examples and
dontcares are duplicated.

Strictly speaking, the counts attached to rules when writing the ruleset
should be those encountered during rule generation. However, for unordered
rules, the counts to attach are generated {\it after} rule generation in
a second pass, following the execution policy of splitting an example
with unknown attribute value into {\it equal} fractions for each value
rather than the Laplace-estimated fractions used during rule generation.

\subsection{Testing (Execution) Phase}

When normally executing unordered CN rules without unknowns, for each
rule which fires the class distribution (ie. distribution of training 
examples among classes) attached to the rule is collected. These are then
summed. Thus a training example satisfying two rules with attached
class distributions [8,2] and [0,1] thus has an expected distribution [8,3]
which results in C1 being predicted, or C1:C2 = 8/11:3/11 if probabilistic
classification is desired. The built-in rule executer follows the first
strategy (the example is classed simply C1).

With unordered CN rules, an attribute test whose value is unknown in the
training example causes the example to be `split'. If the attribute has
three values, 1/3 of the example is deemed to have passed the test and
thus the final class distribution is weighted by 1/3 when collected. A
similar rule later will again cause 1/3 of the example to pass the test.
A don't care value is always deemed to have passed the attribute test in
full (ie. weight 1). The normalisation of the class counts means that 
an example with a dontcare can only count as a single example during testing,
unlike NewID where it may count as representing several examples.

With ordered CN rules a similar policy is followed, except after a rule
has fired absorbing, say, 1/3 of the testing example, only the remaining
2/3s are send down the remainder of the rule list. The first rule will
cause 1/3 $\times$ 
class distribution to be collected, but a second similar rule will
cause 2/3 $\times$ 1/3 $\times$ class distribution to be collected. Thus 
the `fraction'
of the example gets less and less as it progresses down the rule list.
A don't care value always passes the attribute test in full, and thus
no fractional example remains to propagatete further down the rule list.


\end{document}

